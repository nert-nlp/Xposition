\section{Accompanier}

\shortdef{Entity that another entity is together with.}

Sometimes called \emph{comitative}.

Prototypical prepositions are \p{with}, \p{without}, \p{along\_with}, 
\p{together}, \p{together\_with}, and \p{in\_addition\_to}:
\begin{exe}
  \ex I'll have soup \choices{\p{with}\\\p{without}} salad.
  \ex She'll be \p{with} us in spirit.
\end{exe}
`Togetherness' is a subjective concept that goes beyond proximity; 
contrast \cref{ex:withMom} with \cref{ex:nextToMom}, which 
provide slightly different interpretations of the same spatial scene:
\begin{xexe}
  \ex\label{ex:withMom} The girl is standing \p{with} her mother. (\psst{Accompanier})
  \ex\label{ex:nextToMom} The girl is standing \p{next\_to} her mother. (\psst{Locus})
\end{xexe}

For an ``extra participant'' in an activity, 
where two parties perform the activity together 
(but the nature of the activity would not fundamentally 
change if they each performed it independently), 
a \psst{Co-Agent} construal is used:
\begin{exe}
  \ex Do you want to walk \p{with} me? (\rf{Accompanier}{Co-Agent})
\end{exe}
By contrast, if the nature of the scene fundamentally requires multiple participants, 
simple \psst{Co-Agent} is used. Often there is ambiguity:\footnote{Adding \p{together} 
seems to favor the (b)~readings: \pex{I fought \p{together\_with} them}, \pex{We fought \p{together}} 
can only mean we were on the same side. Contrastive stress can also force one reading: 
\pex{I fought \p*{WITH}{with} them (not \p*{AGAINST}{against} them)}.}
\begin{exe}
  \ex Do you want to talk \p{with} me? 
  \begin{xlist}
    \ex {}[\emph{The reading:} Should we have a conversation?] (\psst{Co-Agent})
    \ex {}[\emph{The reading:} Do you want to join me in talking to a third party?] 
      (\rf{Accompanier}{Co-Agent})
  \end{xlist}
  \ex I fought \p{with} them to reform the regulation.
  \begin{xlist}
    \ex {}[\emph{The reading:} I fought against them.] (\psst{Co-Agent})
    \ex {}[\emph{The reading:} I was on the same side as them.] (\rf{Accompanier}{Co-Agent})
  \end{xlist}
\end{exe}

If the object denotes an item that the governor has on hand in their possession, 
then the construal \rf{Possession}{Accompanier} is used:
\begin{exe}
  \ex I walked in \p{with} an umbrella. (\rf{Possession}{Accompanier})
\end{exe}

\paragraph{X\textsubscript{\emph{i}} \emph{bring}/\emph{take}/\dots~Y \p{with} PRON\textsubscript{\emph{i}}.} 
This construction repeats the subject argument in a \p{with}-PP, 
which is analyzed as \rf{Possessor}{Accompanier} or \psst{Accompanier}
depending on whether the scene involves possession (of something nonvolitional) or not:
\begin{exe}
  \ex\begin{xlist}
    \ex I brought my friend \p{with} me. (\psst{Accompanier}) [emphasizes that the (volitional) friend is accompanying the subject]
    \ex I brought my friend.
  \end{xlist}
  \ex\begin{xlist}
    \ex I brought my backpack \p{with} me. (\rf{Possessor}{Accompanier}) [emphasizes that the (nonvolitional) backpack is in the subject's immediate control]
    \ex I brought my backpack.
  \end{xlist}
\end{exe}

See also: \psst{Instrument}, \psst{Manner}

