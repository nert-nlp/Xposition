\section{Duration}

\shortdef{Indication of \textbf{how long} an event or state lasts
(with reference to an amount of time or 
time period\slash larger event that it spans).}

\begin{exe}
  \ex\label{ex:forDuration} I walked \choices{\p{for}\\\#\p{in}} 20~minutes.
  \ex\label{ex:GoalDuration} I walked to$_{\psst{Goal}}$ the store \choices{\p{in}/\p{within}\\\#\p{for}} 20~minutes. [see \cref{ex:inDuration}]
  \ex\label{ex:ExtentDuration} I walked a mile \choices{\p{in}/\p{within}\\\#\p{for}} 20~minutes.
  \ex\label{ex:AmbigDuration} I mowed the lawn \choices{\p{for}\\\p{in}/\p{within}} an hour.
\end{exe}
Note that the presence of a goal \cref{ex:GoalDuration} or 
extent of an event (\pex{a mile} in \cref{ex:ExtentDuration}) 
can affect the choice \psst{Duration} preposition, blocking \p{for}.
\Cref{ex:AmbigDuration} shows a direct object which can be interpreted 
either as something against which partial progress is made---licensing \p{for} 
and the inference that some of the lawn was not reached---or 
as defining the complete scope of progress, licensing \p{in}/\p{within} 
and the inference that the lawn was covered in its entirety.

The object of a \psst{Duration} preposition can also be a reference event 
or time period used as a yardstick for the extent of the main event:
\begin{exe}
  \ex\label{ex:EventDuration} I walked \p{for} the entire race. [the entire time of the race]
  \ex I walked \choices{\p{throughout}\\\p{through}\\well \p{into}} the night.
  \ex\label{ex:overDuration} The deal was negotiated \p{over} (the course of) a year.
\end{exe}
But \p{over} can also mark a time period that \emph{contains} the main event 
and is larger than it. While the path preposition \p{over} highlights that the 
object of the preposition extends over a period of time, it does not require that 
the main event extend over a period of time:
\begin{exe}
  \ex\label{ex:overTimeDuration} He arrived in town \p{over} the weekend. (\rf{Time}{Duration})
\end{exe}
Note that \p{during} can be substituted for \p{over} in \cref{ex:overTimeDuration} but not \cref{ex:overDuration}.

Some \p{for}-\psst{Duration}s measure the length of the specified event's \emph{result}:
\begin{exe}\ex \begin{xlist}
  \ex John went to the store \p{for} an hour. [he spent an hour at the store, not an hour going there]\footnote{This stands 
in contrast with \pex{John walked to the store \p{for} an hour}, where the most natural reading is that it took an hour to get to the store \citep[p.~230]{chang-98}.}
  \ex John left the party \p{for} an hour. [he spent an hour away from the party before returning]
\end{xlist}\end{exe}

A \psst{Duration} may be a stretch of time in which a simple event is repeated 
iteratively or habitually:
\begin{exe}\ex\begin{xlist}
  \ex I lifted weights \p{for} an hour. [many individual lifting acts collectively lasting an hour]
  \ex I walked to the store \p{for} a year. [over the course of a year, habitually went to the store by walking]
\end{xlist}\end{exe}

See further discussion at \psst{Interval}.

