\section{Direction}

\shortdef{How motion or an object is aimed\slash oriented.}

A \psst{Direction} expresses the orientation of a stationary figure or of a figure's motion.
Prototypical markers\footnote{Known variously as \emph{adverbs}, \emph{particles}, 
and \emph{intransitive prepositions}.}
are \p{away} and \p{back}; \p{up} and \p{down}; 
\p{off}; and \p{out},
provided that no specific \psst{Source} or \psst{Goal} is salient:
\begin{exe}
  \ex The bird flew \choices{\p{up}\\\p{out}\\\p{away}\\\p{off}}.
  \ex I walked \p{over} to where they were sitting.
  \ex The price shot \p{up}.
\end{exe}

In addition, transitive \p{toward(s)}, \p{for}, and \p{at} can 
indicate where something is aimed or directed (but see discussion at \psst{Goal}):
\begin{exe}
  \ex The camera is aimed \p{at} the subject.
  \ex The toddler kicked \p{at} the wall.
\end{exe}

See discussion of \p{away\_from} at \psst{Source}.

\paragraph{Distance.}
\rf{Locus}{Direction} is used for expressions of static distance between two points:
\begin{exe}
  \ex 
    \begin{xlist}
      \ex The mountains are 3~km \choices{\p{away}\\\p{apart}}. (\rf{Locus}{Direction})
      \ex The mountains are 3~km \p{away\_from} our house. (\rf{Locus}{Direction})
    \end{xlist}
\end{exe}
This also applies to distances measured by \emph{travel time} (the amount of time 
is taken to be metonymic for the physical distance):
\begin{exe}
  \ex The mountains are an hour \choices{\p{away}\\\p{apart}}. (\rf{Locus}{Direction})
\end{exe}
Compare \psst{Extent}, which is the length of a path of motion or the amount of change.

\paragraph{Informal direction modifier in location description.}
\begin{exe}
  \ex They live (way) \choices{\p{out} past$_{\text{\rf{Locus}{Path}}}$ the highway.\\
    \p{over} by$_{\psst{Locus}}$ the school} (\rf{Locus}{Direction})
\end{exe}
Cf.~\cref{ex:backIntrans} at \psst{Interval}.

