\section{PartPortion}

\shortdef{A part, portion, subevent, subset, or set element (e.g., an example or exception) 
of some \psst{Whole}.}

Anything directly labeled with \psst{PartPortion} 
is understood to be \textbf{incomplete} relative to the \psst{Whole}.
This includes body parts and partial food ingredients.

Prototypical prepositions include \p{with}, \p{without};
\p{such\_as}, \p{like} for exemplification; 
and \p{but}, \p{except}, \p{except\_for} for exceptions:
\begin{exe}
  \ex \begin{xlist}
    \ex	a car \p{with} a new engine
    \ex	a strategy \p{with} 3 prongs
    \ex	the girl \p{with} flaxen hair
    \ex	a man \p{with} a wooden leg named Smith
    \ex	a valley \p{with} a castle
    \ex	a quintet \p{with} 2 cellos
    \ex	a performance \p{with} a guitar solo
    \ex	a cake \p{with} 3 layers
    \ex	a sandwich \p{with} wheat bread
    \ex	soup \p{with} carrots (in it)
    \ex	a chicken sandwich \p{with} ketchup (on it)
  \end{xlist}
  \ex	Bread \p{without} gluten
\end{exe}
Some can be paraphrased with INCLUDES, but this is not determinative.

\paragraph{Elements and Exceptions.} 
\psst{PartPortion} is used for adpositions marking a member or non-member of a set:
\begin{exe}
  \ex\label{ex:suchAs}	strategies \p{such\_as} divide-and-conquer
  \ex Everyone \p{except}/\p{but} Bob plays trombone.
\end{exe}
Set-membership can be construed as comparison:
\begin{exe}
  \ex strategies \p{like} divide-and-conquer [same reading as \cref{ex:suchAs}]\\ (\rf{PartPortion}{ComparisonRef})
\end{exe}

\paragraph{Diverse Examples.}
In describing a set or whole, a sort of scanning with \p{from}\dots\p{to} can be used indicate diversity or coverage of 
the items/parts:
\begin{exe}
  \ex\label{ex:diverserange} Everyone \p{from}$_{\text{\rf{PartPortion}{Source}}}$ the peasants 
  \p{to}$_{\text{\rf{PartPortion}{Goal}}}$ the lord and lady gathered for the feast.
\end{exe}

\paragraph{\pex{Start \p{with}}, \pex{end \p{with}}, etc.}
Along similar lines as \cref{ex:diverserange}, \p{with} can be used with 
an aspectual verb to indicate an item in a sequence: 
\pex{start \p{with}}, \pex{continue \p{with}}, \pex{end \p{with}}, and similar.
Here the scene role \psst{PartPortion} applies 
(though note that it is a part with respect to another argument of the verb, 
not the verb itself):
\begin{exe}
  \ex\rf{PartPortion}{Means}:\begin{xlist} 
    \ex My teacher started the lesson \p{with} a quiz.
    \ex The lesson started \p{with} a quiz.
  \end{xlist}
  \ex The meal started \p{with} an appetizer. (\rf{PartPortion}{Instrument})
\end{exe}

\begin{history}
  In v1, instead of this category, there were separate categories 
  \sst{Elements} for set members, \sst{Comparison/Contrast} for exemplification,
  and \sst{Attribute} for other parts (grouped with properties, which are now \psst{Gestalt}).
  (\sst{Superset} was removed along with \sst{Elements}: see \psst{Whole}.)
\end{history}

