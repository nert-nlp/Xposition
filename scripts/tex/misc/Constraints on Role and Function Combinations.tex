\section{Constraints on Role and Function Combinations}
\label{sec:constraints}

The present scheme emerged out of extensive descriptive work with corpus data. 
Given the abundance of rare preposition usages, this document does not claim 
to cover every possible role\slash function combination for English, 
let alone other languages. 
Below are the few categorical restrictions that seem warranted for English.

\subsection{Supersenses that are purely abstract}

\psst{Participant}, \psst{Configuration}, and \psst{Temporal} are intended only to 
organize subtrees of the hierarchy, and not to be used directly. 

\subsection{Supersenses that cannot serve as functions}

\textbf{\psst{Experiencer}, \psst{Stimulus}, \psst{Originator}, \psst{Recipient}, \psst{SocialRel}, and \psst{OrgRole} 
can only serve as scene roles in English.} 
Though scenes of perception, transfer, and interpersonal\slash organizational relationships 
are fundamental in language, they always seem to exploit construals from other domains 
(motion, causation, possession, and so forth), at least insofar as 
English preposition\slash case marking is concerned. 

For example, \cref{ex:RecGoal} is clearly \psst{Recipient} at the scene level---Sam 
acquires possession of the box---but also 
fits the criteria for \psst{Goal} because Sam is an endpoint of motion 
(and \p{to} frequently marks \psst{Goal}s that are not \psst{Recipient}s). 
\Cref{ex:RecAgent} and \cref{ex:RecPoss} reflect \rf{Recipient}{Agent} and 
\rf{Recipient}{Gestalt} construals, respectively. 
\begin{xexe}
  \ex\label{ex:RecGoal} Give the box \p{to} Sam. (\rf{Recipient}{Goal})
  \ex\label{ex:RecAgent} the box received \p{by} Sam (\rf{Recipient}{Agent})
  \ex\label{ex:RecPoss} Sam\p{'s} receipt of the box (\rf{Recipient}{Gestalt})
\end{xexe}
Though the \psst{Goal} construal is arguably the most canonical expression of \psst{Recipient},
there is no preposition with a primary meaning of \psst{Recipient} independent of one of these other domains.

\textbf{Additional constraints on functions arise in the context of specific 
constructions (\cref{sec:cxns}).} For instance,
\begin{itemize}
  \item the s-genitive requires either \psst{Possessor} or \psst{Gestalt} as its function (\cref{sec:genitives})
  \item passive \p{by} requires \psst{Agent} or \psst{Causer} as its function (\cref{sec:passives})
\end{itemize}

\subsection{Supersenses that cannot serve as roles}

In the present scheme, there are no supersenses that are restricted to serving as functions.

\subsection{No temporal-locational construals}\label{sec:temploc}

\textbf{Temporal prepositions never occur with a function of \psst{Locus}, \psst{Path}, or \psst{Extent}.}

Languages routinely borrow from spatial language to describe time, 
and spatial cognition may underlie temporal cognition \citep[e.g.,][]{lakoff-80,nunez-06,casasanto-08}.
A liberal use of construal would treat \pex{arriving \p{in} the afternoon} as \rf{Time}{Locus}, 
\pex{sleeping \p{through} the night} as \rf{Duration}{Path}, 
\pex{running \p{for} 20~minutes} as \rf{Duration}{Extent}, and so forth.
However, for simplicity and practicality, we elect not to annotate \psst{Locus}, \psst{Path}, or \psst{Extent} 
construals on ordinary temporal adpositions. Thus:
\begin{xexe}
  \ex arriving \p{in} the afternoon (\psst{Time})
  \ex sleeping \p{through} the night (\psst{Duration})
  \ex running \p{for} 20~minutes (\psst{Duration})
\end{xexe}
\rf{Time}{Direction} is possible, however, as are other atemporal functions:
\begin{xexe}
  \ex Schedule the appointment \p{for} Monday. (\rf{Time}{Direction})
  \ex January \p{of} last year (\rf{Time}{Whole})
  \ex Will you attend Saturday\p{'s} class? (\rf{Time}{Gestalt})
  \ex It took a year\p{'s} work to finish the book. (\rf{Duration}{Gestalt})
\end{xexe}

Note that the above is qualified to `ordinary temporal adpositions'. 
\textbf{When the first argument of a comparative construction is marked with \p{as}, 
the function is always \psst{Extent}, even if the scene role is temporal.} 
See \cref{sec:as-as}.

\subsection{Construals where the function supersense is an ancestor or descendant of the role supersense}

Ordinarily, if a construal holds between two (distinct) supersenses, these are from different branches of the hierarchy.
In a few cases, however, one is the ancestor of the other.

\paragraph{Role is ancestor of function.}
\begin{itemize}
  \item Setting events or situations with a salient spatial metaphor are \rf{Circumstance}{Locus} or \rf{Circumstance}{Path}.
  \item Fictive motion (the extension of a normally dynamic preposition to a static spatial scene) 
  can warrant \rf{Locus}{Goal} or \rf{Locus}{Source}, as discussed under \psst{Locus}.
  \item Complete contents of containers are \rf{Characteristic}{Stuff}.
\end{itemize}

\paragraph{Function is ancestor of role.}
\begin{itemize}
  \item Some s-genitives are annotated as \rf{Whole}{Gestalt}: see \cref{sec:genitives}.
  \item When a locative PP is coerced to a goal, as with \emph{put}, \rf{Goal}{Locus} is used.
\end{itemize}



