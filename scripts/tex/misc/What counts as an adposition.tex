\section{What counts as an adposition?}

``Adposition'' is the cover term for prepositions and postpositions. 
Briefly, we consider an affix, word, or multiword expression to be adpositional if it:
\begin{itemize}
  \item mediates a semantically asymmetric figure--ground relation between two concepts, and
  \item is a grammatical item that can mark an NP. 
  We annotate \emph{tokens} of these items even where they mark clauses (as a subordinator) 
  or are intransitive.\footnote{Usually a coordinating conjunction, \p{but}  
  only receives a supersense when it is prepositional, as described 
  under \psst{PartPortion}.}
  We also include always-intransitive grammatical items whose core meaning is spatial and highly schematic, 
  like \p{together}, \p{apart}, and \p{away}.
  % \item Is not a differential object marker (e.g., Hebrew \p{'et}, which marks direct objects if and only if 
  % they are definite).
\end{itemize}
% \ab{differential object markers are ignored because they don't have any lexical semantic content, is that the criterion?}
% 
% \nss{What about a word that matches the above criteria where it is used as an intransitive 
% predicate, e.g. \pex{She is \p{out}/\p{away}}?}\ab{I think that is an adverbial usage instead of adpositional.}

Inspired by \citet{cgel}, the above criteria are broad enough to include 
a use of a word like \p{before} whether it takes an NP complement, 
takes a clausal complement (traditionally considered a subordinating conjunction), 
or is intransitive (traditionally considered an adverb):
\begin{xexe}
  \ex It rained \p{before} the party. [NP complement]
  \ex It rained \p{before} the party started. [clausal complement]
  \ex It rained \p{before}. [intransitive]
\end{xexe}

Even though they are not technically adpositions, 
we also apply adposition supersenses to possessive case marking 
(the clitic \p{'s} and possessive pronouns),
and some uses of the infinitive marker \p{to}, as detailed in \cref{sec:cxns}.

