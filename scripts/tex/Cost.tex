\section{Cost}

\shortdef{An amount (typically of money) that is linked to an item or service 
that it pays for\slash could pay for, or given as the amount earned or owed.} 

The governor may be an explicit commercial scenario:
\begin{exe}
  %\ex I paid/owed John \$10 for the book. %#nonprep
  \ex I \choices{bought\\sold} the book \p{for} \$10.
  \ex I got a refund \p{of} \$10.
  \ex\rf{Cost}{Locus}: \begin{xlist}
    \ex The book is \choices{priced\\valued} \p{at} \$10.
    \ex I bought it \p{at} a great price/rate.
  \end{xlist}
\end{exe}
Or the \psst{Cost} may be specified as an adjunct with a non-commerical governor:
\begin{exe}
  \ex You can ride the bus \p{for} \choices{free\\\$1}.
\end{exe}
\psst{Cost} is \emph{not} used with general scenes of possession or transfer, 
even if the thing possessed or transferred happens to be an amount of money:
\begin{exe}
  \ex I bestowed the winner \p{with} \$100. (\psst{Co-Theme})
\end{exe}

\begin{history}
  This category was not present in v1, which had the broader category \sst{Value}. 
  VerbNet \citep{verbnet,palmer-17} has a similar category called \sst{Asset}; we chose the name 
  \psst{Cost} to emphasize that it describes a relation rather than an entity type 
  (it does not apply to money with a verb like \pex{possess} or \pex{transfer}, 
  for instance).
\end{history}

